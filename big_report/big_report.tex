\documentclass{jarticle}
\usepackage{listings, jvlisting}

\title{自然言語処理の確立的生成過程}
\author{苅込凌太郎}
\date{\today}

\begin{document}
\maketitle

\section{はじめに}
今回の課題では、英語の文章を読み込み、第1次近似と第2次近似、第3次近似の3つの方法で文字列の生成を行った。
文字列生成の際の確率計算は簡便法を使用した。

\section{作成したソースコードの説明}
\begin{lstlisting}
char alphabet_blank[27];

char alphabet_2nd[27][27];
char sentence_2nd[27][1000000];
int  sentence_2nd_number = 0;
int  alphabet_2nd_counter[27][27];
int  chr_2nd_counter[27];
int  blank_2nd_counter[27];

char alphabet_3rd[27][27][27];
char sentence_3rd[27][27][1000000];
int  sentence_3rd_number = 0;
int  alphabet_3rd_counter[27][27][27];
int  chr_3rd_counter[27][27];
int  blank_3rd_counter[27][27];

int  chr_limit_counter = 0;
int  alphabet_1st_number;
int  alphabet_2nd_number;

int  alphabet_low_counter[26];
char alphabet_low[26];
int  alphabet_upp_counter[26];
char alphabet_upp[26];
int  alphabet_number = 0;
char sentence[1000000];
int  sentence_NUMBER = 0;
char chr_output;
char sentence_output[1000000];
int  sentence_output_number = 0;
char sentence_original[1000000];
int  sentence_number = 0;
int  chr_counter = 0;
int  chrblank_counter = 0;
int  blank_counter = 0;
int  blank_flag = 0;
int  indection_flag = 0;

int  limit_output = 100;
int  alphabet_1st_number;
int  alphabet_2nd_number;
\end{lstlisting}
\end{document}
